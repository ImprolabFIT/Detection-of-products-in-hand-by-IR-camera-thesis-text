\begin{conclusion}
Cílem práce bylo vytvoření algoritmu pro detekci zboží v~ruce zákazníka pomocí snímků z~termokamery. K~tomu bylo nutné se seznámit s~technologií termokamer a oborem termografie. V~rešeršní části byly nabyty znalosti pro návrh vlastního řešení, které následně bylo s~jejich pomocí implementováno. Na řešení úlohy bylo nahlíženo různými způsoby a byly navrhnuty celkem čtyři různé algoritmy. Implementovány jsou dva z~nich a jsou k~nalezení v~přiložené aplikaci. Vytvořená aplikace poskytuje uživatelské rozhraní pro snadnou práci s~daty a konfiguraci podstatných parametrů aplikovaných algoritmů. Pro vyhodnocení výsledků práce je použito řešení s~využitím algoritmu dynamického odčítání pozadí. K~vyhodnocování se využívají vlastnoručně naměřená data z~Laboratoře pro zpracování obrazu na FIT ČVUT s~pomocí termokamery FLIR A65 umístěnou nad regálem. Na základě výsledků algoritmu je provedena diskuze, kde jsou uvedeny souhrnné informace o~finálním řešení a jeho možné budoucí vylepšení.

    

\end{conclusion}