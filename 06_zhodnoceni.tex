\chapter{Zhodnocení}
Práce se nakonec ukázala o~něco složitější, než se na začátku zdálo. Přes všechny možné nástrahy od poruchy původní kamery až nestabilitu obrazu se nakonec povedlo najít rozumné řešení, které vzhledem k~dostupným prostředkům poskytuje uspokojivé výsledky. 

\section{Využitá technika}
Během práce se bohužel vyskytla závada na původní kameře (25 mm) a bylo nutné využít záložní kameru (13 mm) s~jinou ohniskovou vzdáleností a horší snímkovací frekvencí. Tato kamera (13 mm) měla širší úhel záběru, takže ji bylo nutné více přiblížit k~regálu. Původní kamera (25 mm) byla pro tuto úlohu vhodnější, protože delší ohnisková vzdálenost přispěla k~tomu, že kolem regálu nebyl snímán zbytečně velký prostor. Jedinou nevýhodou delší ohniskové vzdálenosti je, že v~snímané oblasti regálu byla menší hloubka ostrosti, než při snímání s~kamerou (13 mm). 

7.5 snímků za vteřinu u~náhradní kamery (13 mm) je obecně velmi málo, ale i přesto se dalo při rozšíření oblasti zájmu dosáhnout poměrně dobrých výsledků. V~případě že by bylo vyhodnocování provedeno na funkční kameře (25 mm), byly by výsledky lepší, protože díky vyšší snímkovací frekvenci se algoritmus zvládne rychleji adaptovat na změnu prostředí. Dostačující by byla i kamera se snímkovací frekvencí 15 Hz.

\section{Navržené řešení}
Finální navržený algoritmus se i přes nestabilitu obrazu z~kamery jeví, proti ostatním navrženým řešením, jako robustní. Je schopný se poměrně rychle adaptovat na změny prostředí, která vznikají v~malé míře při postupném zahřívání detektoru a ve velké míře při kalibrací NUC.

Stávající řešení by bylo možné ještě různými způsoby vylepšovat, ale je nutné se zamyslet nad tím, zda se jedná opravdu o~nejlepší možný způsob řešení. Adekvátním vylepšením by bylo zaměřit se na stabilizování snímané scény. Jedním z faktorů, které by toto vylepšení ovlivnilo je algoritmus, pro co nejlepší stanovení hodnot $min$ a $max$ použitých při normalizaci snímku. V~tuto chvíli jsou tyto hodnoty nastavovány ručně. Ruční nastavení poskytuje mnohem lepší výsledky, než normalizace pomocí nalezení minimální a maximální teploty v~snímku. 

Vylepšení, které by přineslo největší přínos je zaměřit se na více obrazových zdrojů. Termokamera se hodí perfektně pro segmentaci ruky, ale segmentovat pomocí termokamery předměty, které mají velmi podobné teploty jako pozadí, se ukázalo jako velmi obtížné. V~\ref{section:combining_rgb_and_thermal} bylo navrženo řešení využívající viditelný a termovizní snímek, ale nakonec nebylo implementováno kvůli vzniklým problémům se synchronizací.

\section{Implementace řešení}
Vytvořená aplikace s~algoritmy detekce slouží též jako klient pro příjem snímků z~aplikace eBUS Player. Mezi přenosem dat těchto dvou aplikací vzniká jen minimální zpoždění a je možné nad daty pracovat téměř v~reálném čase. Toto bylo bráno v~potaz při implementaci jednotlivých metod, které jsou maximálně optimalizovány. Případným rozšířením o~klasifikátor lze tedy ihned vyhodnocovat. Aplikace je též připravena na příjem více datových zdrojů, jako je například obraz z~RGB kamery.

Aplikace má vytvořené uživatelské rozhraní s~velmi dobrou správou nastavení parametrů algoritmů. Návrh aplikace je smysluplný, podstatné části jsou komentovány a je tedy snadné ji dále rozšiřovat.

\section{Výsledky řešení}
Výsledky byly lehce ovlivněny tím, že mezi manipulací se zbožím nebyly nechávány dostatečné přestávky. Tyto nedostatečné pauzy způsobily, že při změně obrazového prostředí (NUC, teplotní drift) nebyl získán dostatečný počet snímků pro adaptaci uloženého pozadí. To způsobuje zhoršení výsledků operace získání popředí pozadí, která na scéně zahrne větší oblast popředí než reálně existuje a prázdná ruka pak může být klasifikována jako ruka se zbožím. Tento problém byl bohužel zjištěn až v~poslední fázi při vyhodnocování, kdy už byla nasnímána všechna data.

Klasifikace probíhala pomocí nástroje RapidMiner a bylo vyzkoušeno mnoho klasifikátorů a různých nastavení. Ke klasifikaci bylo využito 20 atributů zaměřených na vlastnosti kontur. V~případě že by byl k~dispozici stabilní obrazový výstup, dávalo by smysl přidat atributy zaměřující se na vlastnosti obrazu. Přidání dalších užitečných atributů by výsledky klasifikace zlepšilo. 

K~dispozici byl poměrně skromný data set s~přibližně 1200 anotovanými snímky. Při využití většího datasetu by byl vytvořen robustnější model. Klasifikovány byly snímky pouze z~kamery (13 mm) a celková přesnost klasifikace se pohybovala v rozmezí 85 - 88\% v závislosti na využitém klasifikátoru.

\section{Budoucí využití}
V~budoucnu by bylo možné řešení ještě dále rozvíjet, protože místa pro zlepšení stále existují. Jedno z~budoucích témat by mohlo být například detekce pomocí termokamery a kamery Microsoft Kinect s~využitím více obrazových zdrojů. Kombinací obrazových zdrojů by bylo možné stávající řešení zdokonalit, protože každé spektrum přináší o~snímaném obrazu nové informace. Bohužel kombinace obrazů z~různých zdrojů se ukázalo jako velmi obtížné téma a přináší s~sebou spoustu dalších komplikací. Toto téma by se tedy převážně hodilo pro budoucí diplomovou práci.